\documentclass[10pt]{article}
\usepackage{tabulary}
\usepackage{empheq}
\usepackage{hyperref}
\usepackage[margin=1.0in]{geometry} 
\usepackage{amsmath, mathtools}
\usepackage{parskip}
\providecommand{\e}[1]{\ensuremath{\times 10^{#1}}}
\newcommand{\dx}[1]{\mathrm{d}{#1}}
\usepackage{color}
\newcommand{\hilight}[1]{\colorbox{yellow}{#1}}


\begin{document}

\title{FABLE Documentation for RDCEP Website}
\maketitle

%%%%%%%%%%%%%%%%%%%%%%%%%%%%%%%%%%%%%%%%%%%%%%%%%%%%%%%%%%
%%%%%%%%%%%%%%%%%%%%%%%%%%%%%%%%%%%%%%%%%%%%%%%%%%%%%%%%%%

\section{Home Page}
\paragraph{Layout}
This is the first section of the page. The image of the Allocation of Land graph should be to the right of the introduction. We should move the "Limitations" section to the documentation tab. 
\subsection{Introduction}
Competition for land use over time is projected to increase. Changes in demand for energy, forestry products, and food are driven by expected increases in population and changing diets over time, especially in the developing world. These issues are compounded by the uncertainty of agricultural and forestry productivity due to climate change. As such, research has begun to focus more intensely on the optimal allocation of land across sectors and time.\\
FABLE differs from other land use models by integrating components of interest from several research communities. It seeks to address the gap between two land use models. The first are indirect land use change models (iLUCs) that assess short term costs of substituting biofuel crops in areas of highly productive agriculture, therefore forcing food crops to less productive areas or into carbon-rich natural environments. The second are so called integrated assessment models (IAMs) that account for significant greenhouse gas reductions by substituting biofuel production for conventional energy demands.

\subsection{Outline of Model}

The Forest, Agriculture, and Biofuels in a Land use model with Environmental services (FABLE) is a global, intertemporally-consistent optimization model for land use analysis. The model incorporates projections of demands for food, energy and bioenergy, timber and recreation or environmental services in addition to greenhouse gas mitigation targets to solve a constrained welfare maximization problem. The model runs over the period 2005 - 2204, and has accurately predicted developments over the existing history. \\
FABLE takes income, population, wages, oil prices, total factor productivity and other input prices as exogenous. The model is solved by maximizing discounted social welfare across 200 years of global demand for land associated with crop and pasture demand, as well as energy (via conventional liquid fuels and biofuel substitutes), timber and recreational services. \\
To learn more please see the \hilight{Documentation} page.


%%%%%%%%%%%%%%%%%%%%%%%%%%%%%%%%%%%%%%%%%%%%%%%%%%%%%%%%%%
%%%%%%%%%%%%%%%%%%%%%%%%%%%%%%%%%%%%%%%%%%%%%%%%%%%%%%%%%%

\section{FABLE Baseline Tab}
\paragraph{Layout}
This tab has the sectors of the model listed to the left of the page (all sector headings should be links to the documentation page for people to learn more about the model, and formatted as they appear in the documentation page). For the main part of the page, lay out the four main FABLE graphs with spaces underneath for captions. Then have sections underneath the graphs for discussion. 

\subsection{Graphs \& Captions}
Graphs to include:
\paragraph{Allocation of Land} This graph shows the model output for optimal land use in the baseline scenario. In the near term, land dedicated to crops increases. After 2035 cropland decreases due to shifting diets and technological improvements, but is offset by land dedicated to animal feed. Pasture land declines over the course of the simulation, demonstrating the substitution of animal feed for pasture land. Protected land increases substantially over the course of the simulation due to growing weath and corresponding increased value of ecosystem services. Finally, growth in the price of energy result in more land being dedicated to biofuel production. 

\paragraph{Land Based GHG Emissions} This graph shows the gross land based greenhouse gas emissions from their respective sources in the FABLE model, as well as net accumulation. Positive values indicate sources of GHG emissions while negative values indicate abatement. A significant driver of emissions include deforestation in the first quarter of the simulation, which itself is caused by the increased demand for food crops shown in the previous graph. However, increasing access costs of natural land along with declining demand for commercial land results in a decrease in deforestation, reducing emission intensity from this source. Emissions from fertilizer use are relatively stable throughout the modeling timeframe, with increases near the end of the simulation driven by increased livestock feed requirements. Shifting diets drive this increase in meat and dairy demand which simultaneously increases emissions due to enteric fermentation and manure management. As land allocated to forests do not change substantially through the simulation, GHG sequestration in this sector remain stable. The largest growth in GHG sinks appear due to biofuel substitution for petroleum products in the second half of the century, driven by growth in second generation biofuels. 

\paragraph{Land Based Goods and Services} This graph shows per-capital consumption of the goods and services specified in the FABLE model. The consumption of livestock products, timber products, and biofuels grow in absolute terms. This growth is driven by changing preferences, decline in population growth, and rising energy prices. Here the growth in processed livestock product consumption is the most evident, driving land use changes and GHG emissions from the previous two graphs. Declines in the consumption of ecosystem services in the first quarter century are driven by decreases due to deforestation, though greater demand for recreation and growth in protected forests over the remaining years of the simulation minimize further losses. 

\paragraph{Consumption of Biofuels} This graph shows the consumption and share of biofuels in total energy consumption. First generation biofuels grow slightly as energy prices and agricultural lands increase, however the great majority of both consumption and share are due to the adoption of second generation biofuels. The second generation biofuels drive the  increase in land allocated to cellulosic feedstock production in the second half of the simulation, however expansion of land devoted to animal feed limits higher rates of biofuel production and share. Ultimately, share declines even though production continues to expand, due to growing total energy requirements. 



%%%%%%%%%%%%%%%%%%%%%%%%%%%%%%%%%%%%%%%%%%%%%%%%%%%%%%%%%%
%%%%%%%%%%%%%%%%%%%%%%%%%%%%%%%%%%%%%%%%%%%%%%%%%%%%%%%%%%

\section{Run FABLE Tab}
\paragraph{Layout}
Have the same model sectors on the left of the page, but with places underneath each sector for variable input (suggestions listed below). We'll have the plots (with time slider capability), but for now let's leave this TBD. 


%%%%%%%%%%%%%%%%%%%%%%%%%%%%%%%%%%%%%%%%%%%%%%%%%%%%%%%%%%
%%%%%%%%%%%%%%%%%%%%%%%%%%%%%%%%%%%%%%%%%%%%%%%%%%%%%%%%%%

\section{Glossary Tab}
\paragraph{Layout}
Have model sectors on the left (still same style as in the documentation) with links to the Glossary headings for each sector. The main part of the page has the list of glossary topics, organized by sector. 

\subsection{Glossary}
\paragraph{Resource Use - Land}
\begin{itemize}
\item Protected Land - natural parks, biodiversity reserves, and other types of protected forests used to produce ecosystem services
\end{itemize}
\paragraph{Resource Use - Fossil Fuels}
\paragraph{Agrochemical}
\paragraph{Agricultural}
\begin{itemize}
\item CES Function - CES refers to constant elasticity of substitution. A CES production function is a model of production where two or more goods have a constant elasticity of substitution. This means that the ratio of substituting inputs remains constant no matter how much of an input is available.  

\item Imperfect Substitutes - two goods are imperfect substitutes if they both are used to produce another good or service, and output of this final product is greatest when both inputs are used. That is, one input may substitue for the other, but output will be lower because good A is not the same as good B. In FABLE, for example, pasture land and animal feed are imperfect substitutes. Livestock can still live on pasture with less feed, or with more feed and less pasture, but using both will produce the most output. 

\end{itemize}
\paragraph{Livestock Farming}
\paragraph{Food Processing}
\paragraph{Biofuels}
\begin{itemize}
\item 1st Generation Biofuel - Fuel from corn or sugarcane ethanol
\item 2nd Generation Biofuel - Diesel fuels from cellulosic biomass
\end{itemize}
\paragraph{Energy}
\paragraph{Forestry}
\begin{itemize}
\item Vintage - cohort of trees of same age
\end{itemize}
\paragraph{Timber}
\paragraph{Ecosystem Services}
\begin{itemize}
\item Ecosystem Services - economic value of ecosystems based on their continued existence - physical products (e.g., subsistence food and lumber) environmental services (e.g., pollination and nutrition cycling), and non-use goods which are valued purely for their continued existence (e.g., some unobserved biodiversity)
\end{itemize}
\paragraph{Other Goods and Services}
\paragraph{Greenhouse Gas Emissions}
\paragraph{Preferences}
\begin{itemize}
\item Utility - utility is a functional representation of preferences; that is, an equation that describes how an economic agent likes to order different amounts of goods and services. A higher utility with some bundle of goods and services means that the agent prefers that bundle to another that yields lower utility. 
\item AIDADS Model - An implicitly directively additive demand system. This is a model of utility of consumption. For additional information about this model, please refer to the FABLE documentation. 
\item Marginal Budget Shares - A budget share is the percent of a consumer's income that goes towards a particular good or service. The marginal budget share is the amount of an extra unit of income that the consumer choses to spend on additional goods or services to increase their utility. 
\end{itemize}
\paragraph{Welfare}
\begin{itemize}
\item Welfare Function - A welfare function is an equation specifying how a model characterizes economic well-being. It incorporates both the utility of goods and services as well as their costs, and aggregates this value over all economic agents in the society. 
\end{itemize}
\begin{itemize}
\item Scale-Invariant - These are cost functions with constant returns to scale. Per-unit costs remain the same, no matter how much of a quantity is used - eg there are no volume discounts. 
\end{itemize}

%%%%%%%%%%%%%%%%%%%%%%%%%%%%%%%%%%%%%%%%%%%%%%%%%%%%%%%%%%
%%%%%%%%%%%%%%%%%%%%%%%%%%%%%%%%%%%%%%%%%%%%%%%%%%%%%%%%%%

\section{Documentation Tab}
\paragraph{Layout}
Left hand side has a list of links to the sectors of the model, plus a "Math behind the model" and "Resources" link. All of these should link to the main body of the page for content. 

\subsection{Sectors}

\paragraph{Resource Use - Land}

Total land endowment ($\bar{L}$) is fixed, and divided between natural forest lands ($L^N$) and managed commercial lands ($L^M$). Natural lands include natural parks, biodiversity reserves or other protected forests, and used to produce ecosystem services. Within natural land, there exists institutionally protected land ($L^R$), which can never be turned in to commercial land and unmanaged forests ($L^U$). 

\begin{subequations}
\begin{align}
\bar{L} &= L_t^N + L_t^M \\
L_t^N &= L_t^U + L_t^R \\
L_t^M &= L_t^C + L_t^P + L_t^F
\end{align}
\end{subequations}

Unprotected natural land can be either converted to commercial land via deforestation or protected land. Due to the time required to regrow forest cover, once land is deforested in FABLE it cannot return to natural land during the timeframe of the model. There are costs associated with the accessing and conversion of land types; either the cost of building infrastructure to access newly converted commercial land, or the cost of creating new protected land via passing legislation or creating recreational infrastructure. Commercial land is further subdivided in to land allocated to crop land ($L^C$), pasture land ($L^P$), and managed forests ($L^F$). 

\paragraph{Resource Use - Fossil Fuels}

Fossil fuels are used by FABLE in two ways: to supply energy ($x^{f,e}$) and to produce fertilizers ($x^{f,n}$) for use in the agricultural sector. The cost of fossil fuels is exogenous to the model and reflects the costs of extraction, transportation, distribution and any emission control policies. Total supply of fossil fuels is given by:

\begin{equation}
x_t^f = x_t^{f,n} + x_t^{f,e}
\end{equation}

\paragraph{Agrochemical}

The subset of fossil fuels that are used for fertilizer production from the previous module get converted here according to a linear production function that is only dependent on the rate of conversion ($\theta^n$). The non-energy cost of conversion is constant and scale-invariant. 

\begin{equation}
x_t^n = \theta^n x_t^{f,n}
\end{equation}

\paragraph{Agricultural}

This module combines land allocated for crop production and fertilizers from the previous module to yield agricultural output. Output is either food crops (further subset in to food - $x_t^{c, g}$, animal feed - $x_t^{l,g}$, or 1st generation biofuels - $x_t^{b,g}$) or cellulosic feed stocks which are exclusively used to produce 2nd generation biofuels - $x_t^{b,2}$. 

\begin{equation}
x_t^{g,c} = x_t^{c,g} + x_t^{l,g} + x_t^{b,g}
\end{equation}

Land and fertilizers are imperfect substitutes in the production of these two (food crops or cellulosic crops) agricultural products, and output is additionally a function of the agricultural yield index, value share of crop land in agricultural production, and parameters proportional to the elasticity of substitution of agricultural land for fertilizers. Non-land costs of production, such as labor or capital, are predetermined. 

\begin{equation}
x_t^{g,i} = \dfrac{\theta^{g,i}_t}{\Pi_t}\left [\alpha^g \left (L_c^{C,i} \right )^{\rho_g} + \left (1-\alpha^g \right )\left (x_t^n \right )^{\rho_g}\right ]^{\dfrac{1}{\rho_g}} , i = b, c
\end{equation}

\paragraph{Livestock Farming}

Livestock output is a function of pasture land and animal feed, however these inputs are imperfect substitutes. Other parameters that impact output are the livestock technology index ($\theta_t^l$), value share of pasture land in the production of livestock ($\alpha^l$), and a parameter proportional to the elasticity of substitution of pasture land for animal feed. Non-land costs of production, such as labor or capital, are predetermined. 

\begin{equation}
x_t^l = \dfrac{\theta_t^l}{\Pi_t} \left [ \alpha^l \left (L_t^P \right )^{\rho_l} + \left (1-\alpha^l \right) \left (\Pi_t x_t^{l,g} \right )^{\rho_l} \right ]^{\dfrac{1}{\rho_l}}
\end{equation}

\paragraph{Food Processing}

Food processing takes the output from the agricultural and livestock sectors and processes them in to foods consumed ($y_t^g$ and $y_t^l$, respectively) in the final demand function. This sector captures efficiency gains through technological improvements in food production ($\theta_t^{g,y}$ and $\theta_t^{l,y}$), lessening the initial crops and livestock required to satisfy demand. Costs associated with processing both crops and livestock are exogenous and scale-invariant. 

\begin{subequations}
\begin{align}
y_t^g &= \theta_t^{g,y} x_t^{c,g} \\
y_t^l &= \theta_t^{l,y} x_t^l
\end{align}
\end{subequations}

\paragraph{Biofuels}

Food crops not used to satisfy caloric demands are used to produce 1st generation biofuels. Production of these biofuels follow linearly from the amount of remaining food ($x^{b,g}$) according to a technology index ($\theta_t^{b,1}$), which varies exogenously over time. 

\begin{equation}
x_t^{b,1} = \theta_t^{b,1}x^{b,g}
\end{equation}

The sector also converts cellulosic feedstocks in to second generation biofuels. Second generation biofuels are expected to grow in market share according to an S-shaped function due to rate of adoption and availability and cost of capital and technology. The production function of 2nd generation biofuels accounts for the declining impact of fixed inputs according to the rate of factor-specific technological progress in addition to the agricultural yield of cellulosic crops. Other parameters of the production function are  a technology factor, the value share of the fixed factor in 2nd generation biofuel production, and a parameter related to the elasticity of substitution of technology fixed factor for cellulosic feed stocks. Costs associated with the production of biofuels are exogenous and scale-invariant. 

\begin{equation}
x_t^{b,2} = \theta^{b,2} \left [ \left (\alpha^b \right )^{\theta_t^\phi} \left(\phi \right )^{\rho_b} + \left (1-\alpha^b \right ) \left (x_t^{g,b} \right )^{\rho_b} \right ]^{\dfrac{1}{\rho^b}}
\end{equation}

\paragraph{Energy}

The demand for energy is met by first and second generation biofuels in addition to fossil fuels. FABLE assumes that second generation biofuels are perfect substitutes for fossil fuels, while first generation biofuels must be mixed with petroleum products in certain ratios and are therefore imperfect subsitutes. Other parameters influencing supply include the efficiency of energy production, value share of first generation biofuels at the initial time, and a parameter proportional to the elasticity of substitution of first generation biofuels. 

\begin{equation}
y_t^e = \theta_t^e \left ( \alpha^e \left (x_t^{b,1} \right )^{\rho_e} + \left (1-\alpha^e \right ) \left (\dfrac{x_t^{f,e}}{\Pi_t} + x_t^{b,2} \right )^{\rho_e} \right )^{\dfrac{1}{\rho_e}}
\end{equation}

The cost of energy production are simply the sum of the non-land cost of fossil fuels and biofuels from the previous sectors.

\begin{equation}
c_t^e = \sum \limits_i c^{b,1} + c_t^f , i = 1, 2
\end{equation}

\paragraph{Forestry}

Each hectare of managed forest has an average vintage of tree, which can then be either planted, harvested, or left to mature for another period. If the area is harvested, a portion of the mass of trees will yield raw timber according to a monotonically increasing function where higher vintages have higher yields. However, trees cannot be harvested until they reach a minimum age, and trees that have fully matured do not grow further and remain at the maximum vintage ($v_{max}$) until harvested. Costs for harvesting a ton of forest product are scale-invariant and constant across vintages. Since older forests have higher yields, however, cost is a decreasing function of output. Planting a hectare of forest also has a constant, scale invariant cost.

\begin{equation}
L_t^F = \sum \limits_{v=1}^{v_{max}} L_{v,t}^F
\end{equation}

\begin{subequations}
\begin{align}
L_{v+1, t+1}^F &= L_{v,t}^F - \Delta L_{v,t}^{F,H} , v < v_{max} - 1 \\
L_{v_{max}, t+1}^F &= L_{v_{max}, t}^F - \Delta L_{v+{max}, t}^{F, H} + L_{v_{max}-1, t}^F - \Delta L_{v_{max}-1, t}^{F,H} \\
L_{1, t+1}^F &= \Delta L_t^{F,P}
\end{align}
\end{subequations}

\begin{equation}
x_t^w = \sum \limits_{v=1}^{v_{max}} \dfrac{\theta_{v,t}^w}{\Pi_t} \Delta L_{v,t}^{F,H}
\end{equation}



\paragraph{Timber}

Timber processing yield is a function of the amount of timber harvested in that year and the technology index of timber processing ($\theta_t^{w,y}$), which measures efficiency gains and quality improvements. Costs again are exogenous and scale-invariant. 

\begin{equation}
y_t^w = \theta_t^{w,y} x_t^w
\end{equation}


\paragraph{Ecosystem Services}

Ecosystem services are difficult to quantify, both because it is hard to define what constitute these services and because it is difficult to assign a value to many of the services. FABLE calculates the per capita output of ecosystem services to be proportional to a CES production function of all the natural and managed land types such that managed and natural lands are imperfect substitutes for one another. Parameters modifying this relationship include the technology index of ecosystem service production, value shares of crop, pasture, and managed forest lands in production of ecosystem services, and a parameter proportional to the elasticity of substitution of different types of land in production of ecosystem services. The cost of producing ecosystem services is just the maintenance and infrastructure expenditures per hectare of reserved natural land. All other production costs - for agricultural, managed forests and unmanaged lands - are zero. All costs in this module are exogenous and scale-invariant.  

\begin{equation}
y_t^r = \dfrac{\theta^r}{\Pi_t} \left [ \sum \limits_{i= C, P, F} \alpha^{i,r} \left (L_t^i \right )^{\rho_r} + \left (1- \sum \limits_{i = C, P, F} \alpha^{i, r} \right ) \left (L_t^U + \theta^R L_t^R \right )^{\rho_r} \right ] ^{\dfrac{1}{\rho_r}}
\end{equation}


\paragraph{Other Goods and Services}

In order to complete the demand system which determines welfare, all other goods and services aside from agriculture, energy and timber must be accounted for. Because these goods are not dependent on the land resources in this model we assume their production costs are zero and that production grows exogenously. 


\paragraph{Greenhouse Gas Emissions}

Greenhouse gases come from multiple sources within the FABLE model:
\begin{itemize}
\item combustion of petroleum products
\item conversion of unmanaged and managed forests to agricultural land
\item non-$CO_2$ emissions from the use of nitrogen fertilizers in agricultural products
\item non-$CO_2$ emissions from the livestock sector - both enteric fermentation and manure management
\item net GHG sequestration through forest sinks, including the GHG emissions from harvesting forests
\end{itemize}

FABLE differentiates between emissions resulting from petroleum combustion due to the exogenous nature of the price path (which is explained in the Energy sector). All other GHG sources are endogenous to the model. 

Emissions from petroleum combustion, deforestation and fertilizer use are linearly proportional to the use of fossil fuels and allocation of commercial lands. That is, each ton of oil equivalent of fossil fuel combusted or converted to fertilizer, or hectare of land deforested release a predetermined amount of $CO_2$-equivalent greenhouse gases. Livestock emissions are the sum of emissions per hectare pasture land (from manure left on the land) and emissions per ton of livestock produced (from enteric fermentation). 

Sequestration of GHG from forests is dependent on the vintage of that hectare (younger forests sequester carbon at a faster rate), and is limited to managed forests since unmanaged forests are mainly older tree vintages with limited sequestration potential. Harvesting managed forests release GHG  to a lesser extent than total deforestation due to some permanent sequestration in managed forests. FABLE ignores annual cycles of GHG sequestration of agricultural products as these crops are subsequently harvested and consumed. 

Emission control is managed through a quota constraint such that emissions from deforestation, fertilizer application and forest sequestration cannot exceed the quota. Fossil fuel emission constraints are not included in this quota as any such instruments should be managed through the price of the fuels affecting demand and welfare. Since biofuels provide a renewable alternative to fossil fuels, FABLE credits the emissions quota with the fraction of fossil fuel emissions displaced by biofuels. 

\begin{subequations}
\begin{align}
z_t &= \mu^{f,e} x_t^{f,e} + \mu^{f,n} x_t^{f,n} + z_t^L \\
z_t^L &= \mu^L \Delta L_t^U + \mu^P L_t^P + \mu^n x_t^n + \mu^l x_t^l + (1 - \varphi)\sum \limits_{v=1}^{v_{max}} \mu^h_v \Delta L_{v,t}^{F,H} - \sum \limits_{v=1}^{v_{max}} \mu_v^w L_{v,t}^F \\
z_t^L \leq \bar{z_t}^L &= \theta_t^z \left ( z_t^L - \left ( 1 - \dfrac{\mu^{b,1}}{\mu^x} \right ) x_t^{b,i} \right ) , i = 1, 2
\end{align}
\end{subequations}


\paragraph{Preferences}

Preferences, in the form of a utility function, are represented by the AIDADS (an implicitly directively additive demand system) model, which was chosen for it's flexible treatment of Engel effects, global regularity properties, and its superior performance in other studies of food demand (useful for a land-use model).


\hilight{utility source unclear - this section unfinished}

\paragraph{Welfare}

The goal of the social planner is to maximize welfare, which is the discounted net aggregate surplus plus the bequest value of unmanaged and commercial forests. This is calculated by integrating the marginal valuation of each product and subtracting the costs of accessing or producing each good.


\subsection{Math Behind the Model}

\hilight{TBD - how much math and what format?}
\hilight{data sources listed too?}

For full details of FABLE, please read the FABLE technical documentation:\\
\hilight{link to https://www.gtap.agecon.purdue.edu/resources/download/6605.pdf}

FABLE calculates the optimal allocation of land over time by maximizing a social welfare function according to constraints. The welfare function considers both the value and the cost of each type of land. The constraints, based on population, diet, and energy use, are determined by functions governing each sector described above. 




\subsection{Model Constraints \& Resources}
\paragraph{Constraints}
FABLE is a model and as such, has certain constraints in scope and resolution. The model is not intended for detailed policymaking or land use allocation, but instead seeks an optimal distribution of land use, taking in to account the irreversibility of many land use decisions. This model cannot reflect the impacts of market failures such as imperfect information or poorly defined property rights.

\paragraph{Resources}
\begin{itemize}
\item \href{https://www.gtap.agecon.purdue.edu/resources/download/6605.pdf}{Full FABLE Documentation}
\item \href{https://www.gtap.agecon.purdue.edu/resources/download/6047.pdf}{Competition for Land in the Global Bioeconomy}
\item \href{http://iopscience.iop.org/1748-9326/8/1/014014/pdf/1748-9326_8_1_014014.pdf}{Energy Prices Will Play an Important Role in Determining Global Land Use in the Twenty First Century}
\end{itemize}

\end{document}